%% Generated by Sphinx.
\def\sphinxdocclass{report}
\documentclass[a4paper,12pt,russian]{sphinxmanual}
\ifdefined\pdfpxdimen
   \let\sphinxpxdimen\pdfpxdimen\else\newdimen\sphinxpxdimen
\fi \sphinxpxdimen=.75bp\relax

\PassOptionsToPackage{warn}{textcomp}
\usepackage[utf8]{inputenc}
\ifdefined\DeclareUnicodeCharacter
% support both utf8 and utf8x syntaxes
  \ifdefined\DeclareUnicodeCharacterAsOptional
    \def\sphinxDUC#1{\DeclareUnicodeCharacter{"#1}}
  \else
    \let\sphinxDUC\DeclareUnicodeCharacter
  \fi
  \sphinxDUC{00A0}{\nobreakspace}
  \sphinxDUC{2500}{\sphinxunichar{2500}}
  \sphinxDUC{2502}{\sphinxunichar{2502}}
  \sphinxDUC{2514}{\sphinxunichar{2514}}
  \sphinxDUC{251C}{\sphinxunichar{251C}}
  \sphinxDUC{2572}{\textbackslash}
\fi
\usepackage{cmap}
\usepackage[T1]{fontenc}
\usepackage{amsmath,amssymb,amstext}
\usepackage{babel}




\usepackage[Sonny]{fncychap}
\ChNameVar{\Large\normalfont\sffamily}
\ChTitleVar{\Large\normalfont\sffamily}
\usepackage{sphinx}

\fvset{fontsize=\small}
\usepackage{geometry}


% Include hyperref last.
\usepackage{hyperref}
% Fix anchor placement for figures with captions.
\usepackage{hypcap}% it must be loaded after hyperref.
% Set up styles of URL: it should be placed after hyperref.
\urlstyle{same}

\usepackage{sphinxmessages}
\setcounter{tocdepth}{1}



\title{Sphinx и reStructuredText. Учебный проект}
\date{сент. 10, 2020}
\release{}
\author{fish-train}
\newcommand{\sphinxlogo}{\vbox{}}
\renewcommand{\releasename}{}
\makeindex
\begin{document}

\ifdefined\shorthandoff
  \ifnum\catcode`\=\string=\active\shorthandoff{=}\fi
  \ifnum\catcode`\"=\active\shorthandoff{"}\fi
\fi

\pagestyle{empty}
\sphinxmaketitle
\pagestyle{plain}
\sphinxtableofcontents
\pagestyle{normal}
\phantomsection\label{\detokenize{index::doc}}



\chapter{Цель}
\label{\detokenize{index:id1}}
Познакомиться с Sphinx и reStructuredText.


\chapter{Задачи}
\label{\detokenize{index:id2}}\begin{itemize}
\item {} 
Создать проект Sphinx для документации на языке разметки reStructuredText.

\item {} 
Генерировать документацию в формате HTML.

\item {} 
Генерировать документацию в формате PDF.

\item {} 
Хранить исходники в Git.

\item {} 
Публиковать сайт с документацией на ReadTheDocs.

\end{itemize}


\bigskip\hrule\bigskip



\chapter{Содержание}
\label{\detokenize{index:id3}}

\section{Введение}
\label{\detokenize{start:id1}}\label{\detokenize{start::doc}}

\subsection{Установка Sphinx}
\label{\detokenize{start:sphinx}}
\begin{sphinxVerbatim}[commandchars=\\\{\}]
pip install sphinx
\end{sphinxVerbatim}


\subsection{Ссылки}
\label{\detokenize{start:id2}}

\subsubsection{Sphinx}
\label{\detokenize{start:id3}}\begin{itemize}
\item {} \begin{description}
\item[{\sphinxhref{https://www.sphinx-doc.org/en/master/contents.html}{Документация Sphinx}}] \leavevmode
\end{description}

\item {} \begin{description}
\item[{\sphinxhref{https://pythonhosted.org/an\_example\_pypi\_project/sphinx.html}{Documenting Your Project Using Sphinx}}] \leavevmode
\end{description}

\item {} \begin{description}
\item[{\sphinxhref{https://www.ibm.com/developerworks/ru/library/os-sphinx-documentation/index.html}{Простое и удобное создание документации в Sphinx}}] \leavevmode
\end{description}

\item {} \begin{description}
\item[{\sphinxhref{https://sublime-and-sphinx-guide.readthedocs.io/en/latest/index.html}{Create Documentation with RST, Sphinx, Sublime, and GitHub}}] \leavevmode
\end{description}

\end{itemize}


\subsubsection{reStucturedText}
\label{\detokenize{start:restucturedtext}}\begin{itemize}
\item {} \begin{description}
\item[{\sphinxhref{https://www.sphinx-doc.org/en/master/usage/restructuredtext/index.html}{reStructuredText}}] \leavevmode
\end{description}

\item {} \begin{description}
\item[{\sphinxhref{https://www.devdungeon.com/content/restructuredtext-rst-tutorial-0}{reStructuredText (RST) Tutorial}}] \leavevmode
\end{description}

\end{itemize}


\subsubsection{ReadTheDocs}
\label{\detokenize{start:readthedocs}}\begin{itemize}
\item {} 
\sphinxhref{https://docs.readthedocs.io/en/stable/intro/getting-started-with-sphinx.html}{Getting Started with Sphinx}
\begin{quote}
\end{quote}

\end{itemize}


\section{SublimeText}
\label{\detokenize{editor:sublimetext}}\label{\detokenize{editor::doc}}

\subsection{Установка SublimeText}
\label{\detokenize{editor:id1}}
\sphinxhref{https://www.sublimetext.com/3}{Дистрибутив}


\subsection{Установка плагинов}
\label{\detokenize{editor:id3}}\begin{enumerate}
\sphinxsetlistlabels{\arabic}{enumi}{enumii}{}{.}%
\item {} 
Запустите SublimeText.

\item {} 
Выберите пункт меню \sphinxmenuselection{Tools \(\rightarrow\) Command Palette}.

\item {} 
Укажите \sphinxcode{\sphinxupquote{Install Package}} и выберите \sphinxcode{\sphinxupquote{Package Control: Install Package}}.

\item {} 
Выберите плагин из списка.

\end{enumerate}


\subsection{Плагины}
\label{\detokenize{editor:id4}}

\subsubsection{Syntax Highlighting}
\label{\detokenize{editor:syntax-highlighting}}
Подсветка синтаксиса RestructuredText.

\sphinxhref{https://packagecontrol.io/packages/RestructuredText\%20Improved}{Страница плагина Syntax Highlighting}


\subsubsection{OmniMarkupPreviewer}
\label{\detokenize{editor:omnimarkuppreviewer}}
Превью страницы в браузере. Чтобы открыть превью, используйте горячие клавиши \sphinxcode{\sphinxupquote{Ctrl + Alt + O}}.

\sphinxhref{https://github.com/timonwong/OmniMarkupPreviewer}{Страница плагина OmniMarkupPreviewer}


\subsubsection{sublime\sphinxhyphen{}rst\sphinxhyphen{}completion}
\label{\detokenize{editor:sublime-rst-completion}}
Сниппеты и команды для облегчения написания restructuredText в SublimeText.


\begin{savenotes}\sphinxattablestart
\centering
\begin{tabulary}{\linewidth}[t]{|T|T|T|}
\hline
\sphinxstyletheadfamily 
shortcut
&\sphinxstyletheadfamily 
result
&\sphinxstyletheadfamily 
key binding
\\
\hline
\sphinxcode{\sphinxupquote{h1}}
&
Header level 1
&\\
\hline
\sphinxcode{\sphinxupquote{h2}}
&
Header level 2
&\\
\hline
\sphinxcode{\sphinxupquote{h3}}
&
Header level 3
&\\
\hline
\sphinxcode{\sphinxupquote{e}}
&
emphasis
&
\sphinxcode{\sphinxupquote{ctrl+alt+i}}
\\
\hline
\sphinxcode{\sphinxupquote{se}}
&
strong emphasis (bold)
&
\sphinxcode{\sphinxupquote{ctrl+alt+b}}
\\
\hline
\sphinxcode{\sphinxupquote{lit}} или \sphinxcode{\sphinxupquote{literal}}
&
literal text (inline code)
&
\sphinxcode{\sphinxupquote{ctrl+alt+k}}
\\
\hline
\sphinxcode{\sphinxupquote{list}}
&
unordered list
&\\
\hline
\sphinxcode{\sphinxupquote{listn}}
&
ordered list
&\\
\hline
\sphinxcode{\sphinxupquote{listan}}
&
auto ordered list
&\\
\hline
\sphinxcode{\sphinxupquote{def}}
&
term definition
&\\
\hline
\sphinxcode{\sphinxupquote{code}}
&
code\sphinxhyphen{}block directive
&\\
\hline
\sphinxcode{\sphinxupquote{source}}
&
preformatted (:: block)
&\\
\hline
\sphinxcode{\sphinxupquote{img}}
&
image
&\\
\hline
\sphinxcode{\sphinxupquote{fig}}
&
figure
&\\
\hline
\sphinxcode{\sphinxupquote{table}}
&
simple table
&
\sphinxcode{\sphinxupquote{ctrl+t, Enter}}
\\
\hline
\sphinxcode{\sphinxupquote{link}}
&
refered hyperlink
&\\
\hline
\sphinxcode{\sphinxupquote{linki}}
&
embeded hyperlink
&\\
\hline
\sphinxcode{\sphinxupquote{fn}} или \sphinxcode{\sphinxupquote{cite}}
&
autonumbered footnote or cite
&
\sphinxcode{\sphinxupquote{alt+shift+f}}
\\
\hline
\sphinxcode{\sphinxupquote{quote}}
&
Quotation (epigraph directive)
&\\
\hline
\end{tabulary}
\par
\sphinxattableend\end{savenotes}

Доступны блоки примечаний и предупреждений:
\begin{itemize}
\item {} 
attention

\item {} 
caution

\item {} 
danger

\item {} 
error

\item {} 
hint

\item {} 
important

\item {} 
note

\item {} 
tip

\item {} 
warning

\end{itemize}

\sphinxhref{https://github.com/mgaitan/sublime-rst-completion}{Страница плагина sublime\sphinxhyphen{}rst\sphinxhyphen{}completion}


\subsubsection{terminus}
\label{\detokenize{editor:terminus}}
Встроенный терминал.

\sphinxhref{https://github.com/randy3k/Terminus}{Страница плагина terminus}


\section{Сборка HTML}
\label{\detokenize{html:html}}\label{\detokenize{html::doc}}

\subsection{Сайт с несколькими страницами}
\label{\detokenize{html:id1}}
\begin{sphinxVerbatim}[commandchars=\\\{\}]
make html
\end{sphinxVerbatim}

Выходные файлы размещаются в папке \sphinxcode{\sphinxupquote{build/html/}}.


\subsubsection{Скрипт для локальной публикации сайта}
\label{\detokenize{html:id2}}
\begin{sphinxVerbatim}[commandchars=\\\{\},numbers=left,firstnumber=1,stepnumber=1]
\PYG{c+c1}{\PYGZsh{} coding : utf\PYGZhy{}8}

\PYG{k+kn}{import} \PYG{n+nn}{subprocess}

\PYG{c+c1}{\PYGZsh{} Сборка сайта}
\PYG{k}{def} \PYG{n+nf}{make\PYGZus{}site}\PYG{p}{(}\PYG{p}{)}\PYG{p}{:}
    \PYG{c+c1}{\PYGZsh{} Собрать сайт , перейти в папку сайта, запустить веб\PYGZhy{}сервер}
    \PYG{n}{cmd} \PYG{o}{=} \PYG{l+s+s2}{\PYGZdq{}}\PYG{l+s+s2}{make html \PYGZam{} cd build/html \PYGZam{} python \PYGZhy{}m http.server}\PYG{l+s+s2}{\PYGZdq{}}
    \PYG{c+c1}{\PYGZsh{} Выполнить команду CMD}
    \PYG{n}{subprocess}\PYG{o}{.}\PYG{n}{Popen}\PYG{p}{(}\PYG{n}{cmd}\PYG{p}{,} \PYG{n}{shell} \PYG{o}{=} \PYG{n+nb+bp}{True}\PYG{p}{)}

\PYG{n}{make\PYGZus{}site}\PYG{p}{(}\PYG{p}{)}	\PYG{c+c1}{\PYGZsh{} Собрать сайт}
\end{sphinxVerbatim}


\subsection{Одностраничный сайт}
\label{\detokenize{html:id3}}
\begin{sphinxVerbatim}[commandchars=\\\{\}]
make singlehtml
\end{sphinxVerbatim}

Выходные файлы размещаются в папке \sphinxcode{\sphinxupquote{build/singlehtml/}}.


\subsubsection{Скрипт для локальной публикации одностраничного сайта}
\label{\detokenize{html:id4}}
\begin{sphinxVerbatim}[commandchars=\\\{\},numbers=left,firstnumber=1,stepnumber=1]
\PYG{c+c1}{\PYGZsh{} coding : utf\PYGZhy{}8}

\PYG{k+kn}{import} \PYG{n+nn}{subprocess}

\PYG{c+c1}{\PYGZsh{} Сборка сайта}
\PYG{k}{def} \PYG{n+nf}{make\PYGZus{}spage}\PYG{p}{(}\PYG{p}{)}\PYG{p}{:}
    \PYG{c+c1}{\PYGZsh{} Собрать сайт , перейти в папку сайта, запустить веб\PYGZhy{}сервер}
    \PYG{n}{cmd} \PYG{o}{=} \PYG{l+s+s2}{\PYGZdq{}}\PYG{l+s+s2}{make singlehtml \PYGZam{} cd build/singlehtml \PYGZam{} python \PYGZhy{}m http.server}\PYG{l+s+s2}{\PYGZdq{}}
    \PYG{c+c1}{\PYGZsh{} Выполнить команду CMD}
    \PYG{n}{subprocess}\PYG{o}{.}\PYG{n}{Popen}\PYG{p}{(}\PYG{n}{cmd}\PYG{p}{,} \PYG{n}{shell} \PYG{o}{=} \PYG{n+nb+bp}{True}\PYG{p}{)}

\PYG{n}{make\PYGZus{}spage}\PYG{p}{(}\PYG{p}{)} \PYG{c+c1}{\PYGZsh{} Собрать сайт}
\end{sphinxVerbatim}


\subsection{Ссылки на дополнительные материалы}
\label{\detokenize{html:id5}}\begin{itemize}
\item {} 
\sphinxhref{https://sphinx-ru.readthedocs.io/ru/latest/sphinx.html\#html}{Генерация в формат HTML}

\item {} 
\sphinxhref{https://www.sphinx-doc.org/en/master/usage/theming.html}{HTML}

\item {} 
\sphinxhref{https://www.sphinx-doc.org/en/master/usage/configuration.html\#options-for-html-output}{Options for HTML output}

\item {} 
\sphinxhref{https://www.sphinx-doc.org/en/master/usage/configuration.html\#options-for-single-html-output}{Options for Single HTML output}

\item {} 
\sphinxhref{https://www.sphinx-doc.org/en/master/theming.html}{HTML theming support}

\item {} 
\sphinxhref{https://sphinx-themes.org/}{Sphinx Themes}

\end{itemize}


\section{Сборка PDF}
\label{\detokenize{pdf:pdf}}\label{\detokenize{pdf::doc}}

\subsection{Подготовка к работе}
\label{\detokenize{pdf:id1}}\begin{enumerate}
\sphinxsetlistlabels{\arabic}{enumi}{enumii}{}{.}%
\item {} 
Установите \sphinxhref{https://docs.miktex.org/}{MikTex}.

\item {} 
Установите \sphinxhref{https://www.activestate.com/products/perl/downloads/}{ActiveState Perl}.

\item {} 
Установите пакет latexmk.

\item {} 
Установите пакеты, которые предлагает MikTex.

\end{enumerate}


\subsection{Создание PDF\sphinxhyphen{}файлов}
\label{\detokenize{pdf:id2}}
\begin{sphinxVerbatim}[commandchars=\\\{\}]
make latexpdf
\end{sphinxVerbatim}

Выходные файлы размещаются в папке \sphinxcode{\sphinxupquote{build/latex/}}.


\subsection{Ссылки на дополнительные материалы}
\label{\detokenize{pdf:id3}}\begin{itemize}
\item {} 
\sphinxhref{https://www.sphinx-doc.org/en/master/usage/configuration.html\#latex-options}{Options for LaTeX output}

\item {} 
\sphinxhref{https://www.sphinx-doc.org/en/master/latex.html}{LaTeX customization}

\item {} 
\sphinxhref{https://sphinx-ru.readthedocs.io/ru/latest/sphinx.html\#latex}{Генерация в формат LaTeX}

\item {} 
\sphinxhref{https://www.latextemplates.com/}{LaTeX Templates}

\item {} 
\sphinxhref{https://ru.overleaf.com/gallery/tagged/manual}{Gallery — Technical Manual}

\item {} 
\sphinxhref{https://docs.miktex.org/manual/}{MiKTeX Manual}

\item {} 
\sphinxhref{https://habr.com/ru/post/328182/}{Как сделать генерацию LaTeX и PDF в Sphinx}

\end{itemize}


\section{Хостинг на Read The Docs}
\label{\detokenize{ReadTheDocs:read-the-docs}}\label{\detokenize{ReadTheDocs::doc}}
\sphinxhref{https://readthedocs.org}{Read the Docs} \textendash{} сервис для хранения и публикации документации. На Read The Docs можно импортировать проект документации из сервисов: GitHub, Bitbucket and GitLab.

Чтобы импортировать проект из GitHub:
\begin{enumerate}
\sphinxsetlistlabels{\arabic}{enumi}{enumii}{}{.}%
\item {} 
Подготовьте проект:
\begin{enumerate}
\sphinxsetlistlabels{\alph}{enumii}{enumiii}{}{.}%
\item {} 
Убедитесь, что файлы проекта находятся в папке \sphinxcode{\sphinxupquote{docs}}:

\noindent{\hspace*{\fill}\sphinxincludegraphics{{docs_folder}.png}\hspace*{\fill}}

\item {} 
В файл \sphinxcode{\sphinxupquote{conf.py}} добавьте настройку: \sphinxcode{\sphinxupquote{master\_doc = \textquotesingle{}index\textquotesingle{}}}.

\end{enumerate}

\item {} 
Зарегистрируйтесь на сайте \sphinxhref{https://readthedocs.org}{Read the Docs}.

\item {} 
Свяжите свой аккаунт GitHub с Read the Docs. Подробнее см. \sphinxhref{https://docs.readthedocs.io/en/stable/connected-accounts.html}{Connecting Your Account}

\item {} 
Откройте раздел \sphinxstylestrong{Мои проекты} на Read the Docs.

\noindent{\hspace*{\fill}\sphinxincludegraphics{{my_projects}.png}\hspace*{\fill}}

\item {} 
Нажмите на кнопку \sphinxstylestrong{Импортировать проект}. Подробнее см. \sphinxhref{https://docs.readthedocs.io/en/stable/intro/import-guide.html}{Importing Your Documentation}

\item {} 
Выберите репозиторий с проектом.

\item {} 
Проверьте детали проекта и нажмите на кнопку \sphinxstylestrong{Вперед}:

\noindent{\hspace*{\fill}\sphinxincludegraphics{{details}.png}\hspace*{\fill}}

\item {} 
Укажите дополнительные настройки и нажмите на кнопку \sphinxstylestrong{Завершить}:

\noindent{\hspace*{\fill}\sphinxincludegraphics{{extra}.png}\hspace*{\fill}}

\item {} 
Автоматически запустится сборка проекта:

\noindent{\hspace*{\fill}\sphinxincludegraphics{{building}.png}\hspace*{\fill}}

\end{enumerate}

При следующем изменении проекта в репозитории автоматически запустится сборка сайта на Read The Docs.

\begin{sphinxadmonition}{note}{Примечание:}
Если в проекте используются расширения:
\begin{enumerate}
\sphinxsetlistlabels{\arabic}{enumi}{enumii}{}{.}%
\item {} 
Добавьте в папку \sphinxcode{\sphinxupquote{docs}} файл requirements.txt и укажите в нем список зависимостей. Подробнее см. \sphinxhref{https://pip.pypa.io/en/stable/user\_guide/\#requirements-files}{Requirements Files} и \sphinxhref{https://pip.pypa.io/en/latest/reference/pip\_install/\#requirements-file-format}{Requirements File Format} .

\item {} 
На странице проекта нажмите на кнопку \sphinxstylestrong{Админ} и перейдите в \sphinxstylestrong{Расширенные настройки}.

\item {} 
Укажите путь к файлу requirements.txt и нажмите на кнопку \sphinxstylestrong{Сохранить}:

\end{enumerate}

\noindent{\hspace*{\fill}\sphinxincludegraphics{{requirements}.png}\hspace*{\fill}}
\end{sphinxadmonition}


\sphinxstrong{См.также:}


\sphinxhref{https://sphinx-ru.readthedocs.io/ru/latest/rtd-gh.html\#read-the-docs}{Работа с Read The Docs}

\sphinxhref{https://docs.readthedocs.io/en/stable/features.html}{GETTING STARTED}




\chapter{Indices and tables}
\label{\detokenize{index:indices-and-tables}}\begin{itemize}
\item {} 
\DUrole{xref,std,std-ref}{genindex}

\item {} 
\DUrole{xref,std,std-ref}{search}

\end{itemize}



\renewcommand{\indexname}{Алфавитный указатель}
\printindex
\end{document}